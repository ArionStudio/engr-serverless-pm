\documentclass[11pt, a4paper]{article}
\usepackage[T1]{fontenc}
\usepackage{xurl}
\usepackage{hyperref}
\begin{document}
\title{Research do projektu: Serverless Password Manager}
\author{Adrian Rybaczuk}
\date{29 April 2025}
\maketitle

\tableofcontents

\newpage

\section{Definicje}
\subsection{Serverless}

Serverless w naszym projekcie będzie oznaczać dosłownie brak serwera który bedzie zarządzał danymi.
Jedynym punktem wspólnym który pozwoli na synchronizacje danych przez interenet będzie dostęp do usługi Cloud Storage.

\subsection{Cloud Storage}

Cloud Storage to usługa która dostarczana przez zewnętrznego dostawcę, która pozwala na przechowywanie danych w chmurze.
Nasza aplikacja wstępnie będzie korzystać z usługi Amazon S3.

\subsection{Web Extension}

Web Extension - rozszerzenie przeglądarki, pozwala na rozszerzenie funkcjonalności przeglądarki. 
W naszym projekcie będzie to interfejs dostępu do aplikacji.

\subsection{IndexedDB}

IndexedDB to baza danych w przeglądarce, pozwala na przechowywanie danych w ustrukturyzowany sposób.
Będzie służyć jako lokalna baza danych dla naszej aplikacji.

\newpage

\section{Bezpieczeństwo danych}

\subsection{Punkty dostępu do danych}
Punkty dostępu do danych to miejsca w których można uzyskać dostęp do przechowywanych przez nas danych.
W naszym projekcie będą to:
\begin{itemize}
    \item Interfejs Web Extension
    \item IndexedDB
    \item Cloud Storage
\end{itemize}

\subsection{Bezpieczeństwo dostępu do danych}

Dla naszego interfejsu Web Extension będzie docelowym punktem dostępu do danych dla użytkownika. 
Głównym zabezpieczeniem będzie główne hasło dostępu do danych które będzie szyfrowane i przechowywane w IndexedDB.\\
IndexedDB będzie przechowywać dane zaszyfrowane głównym hasłem dostepu.\\
Cloud Storage będzie przechowywać dane w chmurze także zaszyfrowane. Dodatkową warstwą bezpieczeństwa będą klucze dostępu do tych danych.
Zakładamy tutaj, że providery które będą używane zapewniają odpowiednie bezpieczeństwo przechowywanych danych.
Jednak poza tym założeniem dane wysłane do chmury zawsze będą szyfrowane.

\section{Cloud Storage}
\subsection{Do czego użyjemy}
Cloud Storage chcemy użyć do przechowywania i synchronizacji haseł użytkownika.
Użytkownik powinien wybrać docelowego providera, którego chce użyć do przechowywania haseł.
Po wybraniu providera, powinien podać dane dostępowe do danego providera.

Np. w wypadku S3 powinien podać nazwę bucketa, oraz swoje ID i klucz dostępu.
Następnie powinien podać login i hasło dostępu do danych.
Na podstawie którego zostanie zapisany na urządzeniu lokalnym plik konfiguracyjny dostępu do danego providera.
Aplikacja powinna być napisana w sposób, który pozwoli na dostęp do więcej niż jednego providera lub pozwoli na łatwą implementację nowych providerów.

Aktualnie dostępne providery na rynku to:
\begin{itemize}
    \item AWS S3
    \item Google Cloud 
    \item Azure Blob Storage
\end{itemize}
\href{https://www.digitalocean.com/resources/articles/amazon-s3-alternatives}{Źródło} możemy tu znaleźć inne płatne opcje

Opcje open source:
\begin{itemize}
    \item minio
    \item Storj
    \item SwiftStack
\end{itemize}
\href{https://opensourcealternative.to/alternativesto/amazon-s3}{Źródło}

\subsection{AWS S3}

Po przejrzeniu dostępnych opcji wybrałem AWS S3 jako podstawowy provider.
Główną przyczyną była łatwość konfiguracji dla użytkownika aplikacji.

By tę konfigurację uprościć jak najbardziej, stworzyłem template CloudFormation do konfiguracji S3.
Template znajduje się w folderze \texttt{docs/aws/s3-cognito}.
Zawiera on konfigurację dla:
\begin{itemize}
    \item Bucketu S3 
    \item Identity Pool Cognito
    \item Role IAM z uprawnieniami do S3
\end{itemize}

Dokładny opis użycia template znajduje się w pliku \texttt{docs/aws/s3-cognito/README.md}.

Użycie templatki zapewnia poprawną konfigurację bucketu S3 oraz ustawienie odpowiednich zakresów dostępu do niego.

\newpage

\subsection{Web Crypto API}
Dane: \url{https://developer.mozilla.org/en-US/docs/Web/API/Web_Crypto_API}

Do projektu zamierzam użyć Web Crypto API do szyfrowania danych.

\newpage

\section{Platformy docelowe}
Platformy są wstępnie posortowane po priorytetach.
\subsection{Web Addon}
\subsection{Desktop}
\subsection{Android}
\subsection{iOS}

\section{Cechy}
\subsection{Prostota}
\subsection{Bezpieczeństwo}
\subsection{Przenośność}

\section{Podobne projekty}

\subsection{Mopass}
link: \url{https://phodal.github.io/mopass/}

\subsection{Amazon KMS and DynamoDB}
Case: \url{https://towardsaws.com/diy-serverless-password-manager-using-aws-kms-and-amazon-dynamodb-6a8c3798961f}

Problem z tym jest taki, że zyskujemy 2 warstwy pomiędzy AWS Lambda oraz KMS (w tym możemy użyć S3).

\section{Funkcjonalności}
\begin{enumerate}
    \item Dodawanie haseł
    \item Generowanie haseł
    \item Przeglądanie haseł
    \item Edytowanie haseł
    \item Usuwanie haseł
    \item Importowanie haseł\\
    Przyjmuje plik csv z hasłami
    \item Eksportowanie haseł\\
    Eksportuje hasła do pliku csv po wybraniu providera, oraz zakresu haseł i wpisania hasła dostępu
    \item Synchronizacja haseł\\
    Synchronizacja odbywa się automatycznie, wymaga dostępu do internetu oraz włączonej funkcji synchronizacji w ustawieniach.
    \item Wyszukiwanie haseł
    \item Multifactor authentication\\
    Oprócz hasła do repozytorium, będzie możliwość użycia drugiego czynnika uwierzytelniania TOTP/HOTP
    \item Przenośność\\
    Przy pomocy kodu QR z zaszyfrowanymi danymi dostępu do repozytorium, można przenieść repozytorium na inne urządzenie 
\end{enumerate}

\end{document}
